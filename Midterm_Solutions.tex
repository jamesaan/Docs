\documentclass[12pt]{article}
\usepackage[top = 3cm,bottom=3cm,left=2cm,right=2cm]{geometry}        
\geometry{letterpaper}
\usepackage[parfill]{parskip}  
\usepackage{graphicx}
\usepackage{amssymb}
\usepackage{epstopdf}
\usepackage{listings}
\usepackage{color}
\usepackage{amsmath}
\usepackage{tikz}
\usepackage{mathtools}
\usetikzlibrary{decorations.markings,arrows}
\usepackage{bm}
\usepackage{bbm}
\usepackage{hyperref}
 
 \begin{document}  
 
\baselineskip 2.7ex
\parskip 3.5ex

\pagestyle{myheadings}
\markright{\today \hfill Midterm Solutions \hfill}


%Equations/Entries
\newcommand{\beq}{\begin{equation}}
\newcommand{\bequo}{\begin{quotation}}
\newcommand{\beqa}{\begin{eqnarray}}
\newcommand{\eeq}{\end{equation}}
\newcommand{\leeq}[1]{\label{#1}\end{equation}}
\newcommand{\equo}{\end{quotation}}
\newcommand{\eeqa}{\end{eqnarray}}
\newcommand{\non}{\nonumber}
\newcommand{\mx}{\mbox}
\newcommand{\mxf}[1]{\mbox{\footnotesize{#1}}}
\newcommand{\lb}{\label}
\newcommand{\fr}[1]{(\ref{#1})}
\newtheorem{entry}{}[section]
\newcommand{\bent}[1]{\vspace*{-2cm}\hspace*{-1cm}\begin{entry}\lb{e{#1}}\rm}
\newcommand{\eent}{\end{entry}}
\newcommand{\fre}[1]{{\bf\ref{e{#1}}}}
\newcommand{\Emark}{$\sqcap\hspace{-2.7mm}\sqcup$}
\newcommand{\sEmark}{{\fns $\sqcap\hspace{-2.3mm}\sqcup$}}
\newcommand{\fn}{\footnote}


%Greek Letters
\renewcommand{\a}{\alpha}
\renewcommand{\b}{\beta}
\newcommand{\g}{\gamma}
\newcommand{\G}{\Gamma}
\renewcommand{\d}{\delta}
\renewcommand{\th}{\theta}
\renewcommand{\k}{\kappa}
\newcommand{\Th}{\Theta}
\newcommand{\D}{\Delta}
\newcommand{\e}{\epsilon}
\newcommand{\ep}{\varepsilon}
\newcommand{\s}{\sigma}
\renewcommand{\S}{\Sigma}
\newcommand{\w}{\omega}
\newcommand{\W}{\Omega}
\newcommand{\al}{\alpha}
\newcommand{\bet}{\beta}
\newcommand{\gam}{\gamma}
\newcommand{\lam}{\lambda}
\newcommand{\Lam}{\Lambda}
\newcommand{\eps}{\varepsilon}
\newcommand{\ichi}\sichi
\renewcommand{\ni}{\sni}
\renewcommand{\r}{\rho}
\renewcommand{\t}{\tau}
\newcommand{\ph}{\varphi}
\newcommand{\sichi}{{\mbox{{\footnotesize I}}}}
\newcommand{\sni}{{\mbox{{\footnotesize II}}}}

%color2010/6/9
\newcommand{\red}{\color{red}}
\newcommand{\blue}{\color{blue}}
\newcommand{\green}{\color{green}}
\definecolor{gray}{rgb}{0.5, 0.5, 0.5}
\newcommand{\gray}{\color{gray}}

%Derivatives
\newcommand{\pder}[2]{\frac{\partial {#1}}{\partial {#2}}}
\newcommand{\pdert}[2]{\frac{\partial^2 {#1}}{\partial {#2}^2}}
\newcommand{\fder}[2]{\frac{\delta {#1}}{\delta {#2}}}
\newcommand{\PDD}[3]{\left.\frac{\partial^{2}{#1}}{\partial{#2}^{2}}\right|_{#3}
}
\newcommand{\PD}[3]{\left.\frac{\partial{#1}}{\partial{#2}}\right|_{#3}}
\newcommand{\der}[2]{\frac{d {#1}}{d {#2}}}

\renewcommand{\deg}{^\circ}
\newcommand{\com}{{\bf [C] }}
\newcommand{\cend}{\Emark\[\]\vspace*{-1. cm}}
\newcommand{\x}{\times}

%My commands
\newcommand{\win}{\ddot\smile}
\newcommand{\lose}{\ddot\frown}
\newcommand{\avg}[1]{\left \langle #1 \right \rangle}
\newcommand{\E}[1]{\ensuremath{\times10^{#1}}}
\newcommand{\abs}[1]{\ensuremath{\left | #1 \right |}}
\newcommand{\paren}[1]{\left(#1\right)}
\newcommand{\recip}[1]{\frac{1}{#1}}
\newcommand{\ex}[1]{\mathbb{E}[#1]}
\newcommand{\bprob}[1]{\textbf{#1~---}}
\newcommand{\unitv}[1]{\ensuremath{\mathbf{\hat{e}}_{#1}}}
\newcommand{\goto}{\rightarrow}
\newcommand{\expct}[1]{\mathbb{E}[#1]}
\newcommand{\mtrx}[1]{\begin{matrix}#1\end{matrix}}
\newcommand{\pmtrx}[1]{\paren{\begin{matrix}#1\end{matrix}}}
\newcommand{\cosp}[1]{\cos{\paren{#1}}}
\newcommand{\sinp}[1]{\sin{\paren{#1}}}
\newcommand{\tanp}[1]{\tan{\paren{#1}}}
\newcommand{\half}[1]{\frac{#1}{2}}
\newcommand{\ham}{\mathcal{H}}
\newcommand{\tr}{\mathrm{Tr}}
\newcommand{\bv}[1]{\mathbf{#1}}
\newcommand{\Der}[2]{\frac{d#1}{d#2}}
\renewcommand{\Dot}[2]{\ensuremath{\bv{#1}\cdot\bv{#2}}}
\newcommand{\Cross}[2]{\ensuremath{\bv{#1}\times\bv{#2}}}
\newcommand{\del}{\ensuremath{\partial}}
\newcommand{\R}{\ensuremath{\bv{r-r'}}}
\newcommand{\aR}{\ensuremath{\abs{\R}}}
\newcommand{\br}{\ensuremath{\bv{r}}}
\newcommand{\impl}{\ensuremath{\quad \Rightarrow \quad}}
\renewcommand{\div}[1]{\nabla \cdot \bv{#1}}
\newcommand{\curl}[1]{\nabla \times \bv{#1}}
\newcommand{\lapl}{\nabla^2}
\newcommand{\vint}{\int d^3r}
\newcommand{\oocs}{\recip{c^2}}
\newcommand{\mnfp}[1]{\frac{\mu_0 #1}{4\pi}}
\renewcommand{\iiint}{\int_{-\infty}^{\infty}}
\newcommand{\tpi}[1]{\paren{2\pi}^{#1}}
\newcommand{\ootpi}[1]{\recip{\paren{2\pi}^{#1}}}
\newcommand{\Sig}{\bm{\s}}
\renewcommand{\dag}{^\dagger}

%%%%%%%%%%%%%%%%%%%%%%%

\bprob{1} a) True, \qquad b) 2 Legendre transforms, \qquad c) all of the above

d) $4^{20}$, \qquad e) i, ii, and iii

\hrulefill

\bprob{2a} Total length:
\[ L = aN_a + bN_b, \qquad N = N_a + N_b.\]
Fixing $L$ fixes $N_a$ and $N_b$, but for a given $L$, there are
\[ \Omega = { N \choose N_b} = { N \choose \frac{L-aN}{b-a} }\]
possible microstates.

The value of $N_a$ that maximizes this is $N/2$, so $L = (a+b)N/2$.

\bprob{2b} The entropy at a given $L$ is
\[ S = k_b \log \Omega(L) = k_B \log { N \choose \frac{L-aN}{b-a} }.\]
The Helmholtz free energy is just $-TS$,  because there is no internal energy in this model, we consider only entropic effects.
\[ A = -TS = -k_BT \log { N \choose \frac{L-aN}{b-a} }.\]
If you want, you can go to the canonical ensemble:
\[ Q = \sum_\nu \w_n e^{-\b E_\nu}.\]
Here, $\w_n = \Omega(L)$ and $E_\nu = 0$ for all states $\nu$, so there's no sum over $\nu$ at all, because all the states have the same energy, and our degeneracy is $\Omega(L)$, so the canonical is the same as the microcanonical ensemble here:
\[ A = -k_BT \log Q = -k_BT \log \Omega(L).\]
We use Stirling's approximation on ${N \choose N_b}$ and we get:
\[ A = -k_BT \paren{ N\log N - N_b \log N_b - (N-N_b) \log(N-N_b) }.\]

\bprob{2c} From $dA = -SdT + JdL$, we see that
\[ J = \PD{A}{L}{T}.\]
and we have
\[ \pder{}{L} = \recip{b-a}\pder{}{N_a}.\]
So we get
\[ J = -k_BT \pder{}{N_b} \paren{ - N_b \log N_b + (N-N_b)\log(N-N_b) }.\]
\[ J = \frac{k_BT}{b-a} \log \frac{N_b}{N-N_b}.\]
Notice that when $N_b > N/2$, when we have the polymer extended beyond its equilibrium length, $J > 0$, and when $N_b < N/2$, we have $J<0$. This is the meaning of entropic elasticity.

\hrulefill

\bprob{3a} We're given
\[ \PD{S}{L}{T} < 0.\]
And from our fundamental relation
\[ dE = TdS + JdL \goto dA = dE - d(TS) = -S dT + JdL.\]
From this we can read off a Maxwell relation:
\[ -\PD{S}{L}{T} = \PD{J}{T}{L} > 0.\]

\bprob{3b} We can use the cyclic derivative rule here
\[ \PD{T}{L}{S}\PD{L}{S}{T} \PD{S}{T}{L} = -1.\]
\[ \PD{T}{L}{S} = -\PD{S}{L}{T} \paren{\frac{T}{C_L}} > 0.\]

\hrulefill

\bprob{4a} If we pin $n$ particles to the surface, there are ${N \choose n}$ ways to arrange them on the surface. This also leave $N-n$ particles to roam free in the bulk. The number of ways to arrange \emph{those} particles is ${M \choose N-n}$, so we have
\[ \Omega = {N \choose n} {M \choose N-n}.\]
The entropy is
\[ S = k_B\log \Omega.\]

\bprob{4b} Here, it is \emph{really hard} to work in the canonical ensemble. We stay in the microcanonical ensemble here. So we have
\[ E = -\ep n,\qquad \recip{T} = \pder{S}{E}.\]
\[ \recip{T} = -\recip{\ep} \pder{S}{n}.\]
\[ -\b \ep = \pder{\log \Omega}{n}.\]
So after doing all the Stirling expansion:
\[ \pder{\log \Omega}{n} = \pder{}{n} \paren{ N\log N - n\log n - 2(N-n)\log(N-n) + M\log M  - (M-N+n)\log(M-N+n) }.\]
\[ -\b \ep = \log \paren{ \frac{(N-n)^2}{n(M-N+n)} }.\]
\[ e^{-\b \ep} = \frac{ (N-n)^2}{n(M-N+n)}.\]
It's really hard to invert this, I think, I haven't tried. But it's not very interesting if it is invertible.

\bprob{4c} If $T\goto 0$, then $\b\goto \infty$ and $e^{-\b \ep} \goto 0$, so we have
\[ (N-n)^2 = 0.\]
\[ \frac{n}{N} = 1.\]

\bprob{4d} Now we send $T\goto \infty$, $\b \goto 0$, and $e^{-\b \ep} \goto 1$,
\[ (N-n)^2 = n(M-N+n).\]
\[ \frac{n}{N} = \frac{N}{M+N}.\]

\hrulefill

\bprob{5a} The magnetic degree of freedom is just in the dipole, the orientation of the particle. The ideal gas partition function only cared about momentum and position, so this is a degree of freedom that isn't coupled at all to the ideal gas degrees of freedom, so I'll abbreviate those:
\[ Q = Q_\mathrm{ideal} Q_\mathrm{magnetic} = Q_I Q_M.\]
\[ Q_I = \recip{N!} \paren{\frac{V}{\Lambda^3} }^N.\]
The magnetic degree of freedom is quantized, and it only has six states:
\[ Q_M = e^{\b Bm} + 1 + 1 + 1 + 1 + e^{-\b Bm} = e^{\b Bm} + e^{-\b Bm} + 4.\]

\bprob{5b} Since the partition functions are multiplicatively decoupled, then the free energy separates into two additive terms:
\[ A = A_I + A_M.\]
\[ A = A_I - Nk_BT \log \paren{ e^{\b Bm} + e^{-\b Bm} + 4}.\]

\bprob{5c} Since the magnetic free energy doesn't depend on the volume, it won't contribute to the pressure:
\[ P = -\PD{A}{V}{T} = -\PD{A_I}{V}{T} = \frac{Nk_BT}{V}.\]

\bprob{5d} The internal energy also separates:
\[ E = -\pder{\log Q}{\b} = -\pder{\log Q_I}{\b} - \pder{\log Q_M}{\b}.\]
\[ E = \frac{3}{2}\frac{N}{\b}  - NBm \frac{e^{\b Bm} - e^{-\b Bm}}{e^{\b Bm} + e^{-\b Bm} + 4}.\]



\end{document}

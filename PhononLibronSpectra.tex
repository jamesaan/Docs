\documentclass[12pt]{article}
\usepackage[top = 3cm,bottom=3cm,left=2cm,right=2cm]{geometry}        
\geometry{letterpaper}
\usepackage[parfill]{parskip}  
\usepackage{graphicx}
\usepackage{amssymb}
\usepackage{epstopdf}
\usepackage{listings}
\usepackage{color}
\usepackage{amsmath}
\usepackage{tikz}
\usepackage{mathtools}
\usetikzlibrary{decorations.markings,arrows}
\usepackage{bm}
 
 \begin{document}  
 
\baselineskip 2.7ex
\parskip 3.5ex

\pagestyle{myheadings}
\markright{James Antonaglia \hfill Phonon Libron Spectra \hfill}


%Equations/Entries
\newcommand{\beq}{\begin{equation}}
\newcommand{\bequo}{\begin{quotation}}
\newcommand{\beqa}{\begin{eqnarray}}
\newcommand{\eeq}{\end{equation}}
\newcommand{\leeq}[1]{\label{#1}\end{equation}}
\newcommand{\equo}{\end{quotation}}
\newcommand{\eeqa}{\end{eqnarray}}
\newcommand{\non}{\nonumber}
\newcommand{\mx}{\mbox}
\newcommand{\mxf}[1]{\mbox{\footnotesize{#1}}}
\newcommand{\lb}{\label}
\newcommand{\fr}[1]{(\ref{#1})}
\newtheorem{entry}{}[section]
\newcommand{\bent}[1]{\vspace*{-2cm}\hspace*{-1cm}\begin{entry}\lb{e{#1}}\rm}
\newcommand{\eent}{\end{entry}}
\newcommand{\fre}[1]{{\bf\ref{e{#1}}}}
\newcommand{\Emark}{$\sqcap\hspace{-2.7mm}\sqcup$}
\newcommand{\sEmark}{{\fns $\sqcap\hspace{-2.3mm}\sqcup$}}
\newcommand{\fn}{\footnote}


%Greek Letters
\renewcommand{\a}{\alpha}
\renewcommand{\b}{\beta}
\newcommand{\g}{\gamma}
\newcommand{\G}{\Gamma}
\renewcommand{\d}{\delta}
\renewcommand{\th}{\theta}
\renewcommand{\k}{\kappa}
\newcommand{\Th}{\Theta}
\newcommand{\D}{\Delta}
\newcommand{\e}{\epsilon}
\newcommand{\ep}{\varepsilon}
\newcommand{\s}{\sigma}
\renewcommand{\S}{\Sigma}
\newcommand{\w}{\omega}
\newcommand{\W}{\Omega}
\newcommand{\al}{\alpha}
\newcommand{\bet}{\beta}
\newcommand{\gam}{\gamma}
\newcommand{\lam}{\lambda}
\newcommand{\Lam}{\Lambda}
\newcommand{\eps}{\varepsilon}
\newcommand{\ichi}\sichi
\renewcommand{\ni}{\sni}
\renewcommand{\r}{\rho}
\renewcommand{\t}{\tau}
\newcommand{\ph}{\varphi}
\newcommand{\sichi}{{\mbox{{\footnotesize I}}}}
\newcommand{\sni}{{\mbox{{\footnotesize II}}}}

%color2010/6/9
\newcommand{\red}{\color{red}}
\newcommand{\blue}{\color{blue}}
\newcommand{\green}{\color{green}}
\definecolor{gray}{rgb}{0.5, 0.5, 0.5}
\newcommand{\gray}{\color{gray}}

%Derivatives
\newcommand{\pder}[2]{\frac{\partial {#1}}{\partial {#2}}}
\newcommand{\pdert}[2]{\frac{\partial^2 {#1}}{\partial {#2}^2}}
\newcommand{\fder}[2]{\frac{\delta {#1}}{\delta {#2}}}
\newcommand{\PDD}[3]{\left.\frac{\partial^{2}{#1}}{\partial{#2}^{2}}\right|_{#3}
}
\newcommand{\PD}[3]{\left.\frac{\partial{#1}}{\partial{#2}}\right|_{#3}}
\newcommand{\der}[2]{\frac{d {#1}}{d {#2}}}
\newcommand{\pdder}[3]{\frac{\del^2 {#1}}{\del {#2} \del {#3}}}


\renewcommand{\deg}{^\circ}
\newcommand{\com}{{\bf [C] }}
\newcommand{\cend}{\Emark\[\]\vspace*{-1. cm}}
\newcommand{\x}{\times}

%My commands
\newcommand{\win}{\ddot\smile}
\newcommand{\lose}{\ddot\frown}
\newcommand{\avg}[1]{\left \langle #1 \right \rangle}
\newcommand{\E}[1]{\ensuremath{\times10^{#1}}}
\newcommand{\abs}[1]{\ensuremath{\left | #1 \right |}}
\newcommand{\paren}[1]{\left(#1\right)}
\newcommand{\recip}[1]{\frac{1}{#1}}
\newcommand{\ex}[1]{\mathbb{E}[#1]}
\newcommand{\bprob}[1]{\textbf{#1~---}}
\newcommand{\unitv}[1]{\ensuremath{\mathbf{\hat{e}}_{#1}}}
\newcommand{\goto}{\rightarrow}
\newcommand{\expct}[1]{\mathbb{E}[#1]}
\newcommand{\mtrx}[1]{\begin{matrix}#1\end{matrix}}
\newcommand{\pmtrx}[1]{\paren{\begin{matrix}#1\end{matrix}}}
\newcommand{\cosp}[1]{\cos{\paren{#1}}}
\newcommand{\sinp}[1]{\sin{\paren{#1}}}
\newcommand{\tanp}[1]{\tan{\paren{#1}}}
\newcommand{\half}[1]{\frac{#1}{2}}
\newcommand{\ham}{\mathcal{H}}
\newcommand{\tr}{\mathrm{Tr}}
\newcommand{\bv}[1]{\mathbf{#1}}
\newcommand{\Der}[2]{\frac{d#1}{d#2}}
\renewcommand{\Dot}[2]{\ensuremath{\bv{#1}\cdot\bv{#2}}}
\newcommand{\Cross}[2]{\ensuremath{\bv{#1}\times\bv{#2}}}
\newcommand{\del}{\ensuremath{\partial}}
\newcommand{\R}{\ensuremath{\bv{r-r'}}}
\newcommand{\aR}{\ensuremath{\abs{\R}}}
\newcommand{\br}{\ensuremath{\bv{r}}}
\newcommand{\impl}{\ensuremath{\quad \Rightarrow \quad}}
\renewcommand{\div}[1]{\nabla \cdot \bv{#1}}
\newcommand{\curl}[1]{\nabla \times \bv{#1}}
\newcommand{\lapl}{\nabla^2}
\newcommand{\vint}{\int d^3r}
\newcommand{\oocs}{\recip{c^2}}
\newcommand{\mnfp}[1]{\frac{\mu_0 #1}{4\pi}}
\renewcommand{\iiint}{\int_{-\infty}^{\infty}}
\newcommand{\tpi}[1]{\paren{2\pi}^{#1}}
\newcommand{\ootpi}[1]{\recip{\paren{2\pi}^{#1}}}

%%%%%%%%

\section{Coupled Lattice Fields}
I'm going to write a simple toy model to illustrate the approach to solving a system of equations to find the dispersion relation for some coupled fields on our lattice.
\subsection{1D Chain}
Let's start simple, with a one dimensional chain with some harmonic coupling between some arbitrary fields $a_i$ and $b_i$ that are defined on our one dimensional lattice of lattice spacing $\ell$. The potential energy will be
\[ V = \sum_i \half{1}A (a_i-a_{i+1})^2 + \half{1} B(b_i - b_{i+1})^2 + C(a_i-a_{i+1})(b_i - b_{i+1}) + \half{1} M a_i^2.\]
The first two terms are the coupling between nearest neighboring lattice sites, the third term couples the $a$ fields to the $b$ fields, and the last term we'll see is what will give the $a$ modes a mass, \emph{i.e.} their frequencies don't go to zero at zero wavelength. Let's give the $a$ particles a mass $m$ and the $b$ particles a mass $\mu$. So the Lagrangian looks like
\[ \mathcal{L} = \sum_i \half{1}m \dot{a}_i^2 + \half{1}\mu \dot{b}_i^2 - V.\]
Newton's equations then read:
\[ m\ddot{a}_i = -2Aa_i + Aa_{i+1} +Aa_{i-1} - 2Cb_i +Cb_{i+1} + Cb_{i-1} - Ma_i.\]
\[ m\ddot{b}_i = -2Ca_i + Ca_{i+1} +Ca_{i-1} - 2Bb_i +Bb_{i+1} + Bb_{i-1}.\]
Now we go to Fourier space to turn this difference equation into an algebraic equation. The transformation is:
\[ a_j = \sum_j \tilde{a}(k) e^{-i k(j\ell)},\qquad b_j = \sum_j \tilde{b}(k) e^{-ik(j\ell)}.\]
We put in these transforms in the above difference equations and exploit the orthonormality of the exponential to get equality term by term:
\[ m\ddot{\tilde{a}} = -2A\tilde{a} + A\tilde{a}e^{-ik\ell} + A\tilde{a}e^{ik\ell} - 2C \tilde{b} + C \tilde b e^{-ik\ell} + C\tilde b e^{i k \ell} - M\tilde a.\]
\[ \mu\ddot{\tilde{b}} = -2C\tilde{a} + C\tilde{a}e^{-ik\ell} + C\tilde{a}e^{ik\ell} - 2B \tilde{b} + B \tilde b e^{-ik\ell} + B\tilde b e^{i k \ell}.\]
We then look for time-harmonic solutions such that $\der{^2}{t^2}\goto -\w^2$, and we finally have a matrix equation, setting $\sinp{k\ell/2} = \g$:
\[ \pmtrx{m\w^2 - 4\g^2 A - M & -4\g^2 C \\ - 4\g^2 C & \mu \w^2 - 4\g^2 B} \pmtrx{\tilde a \\ \tilde b} = 0.\]
Now if we wanted to, we could put this in Mathematica and get the eigenmodes and corresponding dispersion relations. But let's just set $k=0$ so $\gamma = 0$:
\[ \pmtrx{m\w^2 - M & 0 \\ 0 & \mu \w^2}\pmtrx{\tilde a \\ \tilde b} = 0.\]
The solution to this eigenvalue problem is $\w^2 = 0$ and $\w^2 = M/m$. The first case has eigenvector $\tilde a = 0$ and $\tilde b = b_0$, so this is a wave with zero frequency and zero wavelength with arbitrary amplitude, which corresponds to a global constant shift of the $b$ field. This doesn't couple to the $a$ field. However, for the other eigenvalue, $\w = \sqrt{M/m}$ is a finite frequency with $k=0$ and $\tilde a = a_0$ and $\tilde b = 0$. This is a global and spatially homogenous oscillation of the $a$ field.

\subsection{$d=n$ Spring Field}
Now let's consider two coupled fields in $n$ dimensions situated on a lattice. Suppose that we don't need a basis to characterize our unit cell (if we did, we'd expect some optical modes), and let the nearest neighbors of a lattice site be given by the set of vectors $\{\bv c\}$. For this situation, we must sum up the potential energies of all the bonds and be careful not to double count, so we should include appropriate $1/2$'s. So if we focus on a lattice site at position $\br$, then the potential energy is:
\[ V = \sum_\br \half{1}\paren{\sum_{\bv c}\recip{2}A(a_\br - a_{\br + \bv c})^2 + \recip{2}B(b_\br - b_{\br + \bv c})^2 + C(a_\br - a_{\br+\bv c})(b_\br - b_{\br+\bv c})} + \half{1}Ma_\br^2.\]
So we once again prescribe some kinetic energy:
\[ \mathcal{L} = \sum_\br \half{1}m \dot{a}_\br^2 + \half{1}\mu \dot{b}_\br^2 - V.\]
\[ m\ddot{a}_\br = -Ma_\br -\recip{2} \sum_{\bv c} A\paren{2 a_\br - a_{\br + \bv c} - a_{\br - \bv c}}-\recip{2}\sum_{\bv c} C\paren{2 b_\br - b_{\br + \bv c} - b_{\br - \bv c}}.\]
\[ \mu\ddot{b}_\br = -\recip{2} \sum_{\bv c} C\paren{2 a_\br - a_{\br + \bv c} - a_{\br - \bv c}}-\recip{2}\sum_{\bv c} B\paren{2 b_\br - b_{\br + \bv c} - b_{\br - \bv c}}.\]
From here, we once again go to Fourier space and the story is very similar. We impose time-harmonic solutions and solve another eigenproblem.
\[ a(\br) = \sum_{\bv k} \tilde{a}_{\bv k} e^{-i \Dot k r},\qquad b(\br) = \sum_{\bv k} \tilde{b}_{\bv k}e^{-i \Dot k r}.\]
\[ -\w^2 m \tilde a = -M \tilde a - \half{1}\sum_{\bv c}A \paren{2 \tilde a - \tilde a e^{-i\Dot k c} - \tilde a e^{i\Dot k c}} - \half{1}\sum_{\bv c}C \paren{2 \tilde b - \tilde b e^{-i\Dot k c} - \tilde b e^{i\Dot k c}}.\]
\[ -\w^2 \mu \tilde b =  - \half{1}\sum_{\bv c} C\paren{2 \tilde a - \tilde a e^{-i\Dot k c} - \tilde a e^{i\Dot k c}} - \half{1}\sum_{\bv c} B\paren{2 \tilde b - \tilde b e^{-i\Dot k c} - \tilde b e^{i\Dot k c}}.\]
\[ -\w^2 m \tilde a = -M \tilde a + 4A\g^2 \tilde a + 4C\g^2 \tilde b.\]
\[ -\w^2 \mu \tilde a = 4C \g^2 \tilde a + 4B \g^2 \tilde b.\]
Here, the story is the same as the $d=1$ case, except instead of $\g = \sinp{k\ell/2}$, we have
\[ \g^2 = \half{1} \sum_{\bv c}\sin^2 \paren{\half{\Dot k c}}.\]
This is comforting, because if we set $\{\bv c\} = \pm \ell \hat{x}$, then $\g = \sin^2\paren{\half{k\ell}}$. So we're back to the simple 1D case. But here, the factor $\g$ is just characteristic of the lattice. For true crystals, there is a neighbor that lives $+\bv c$ away from the site at $\br$ and also at $-\bv c$ from $\br$, so we can use this symmetry to simplify our expression for $\g$, but really this isn't a difficult thing to compute. So we've come to the regular matrix equation in 1D as the solution in $n$D.

Let's calculate $\g^2$ for a cubic lattice:
\[ \{\bv c\} = \{\pm \hat x, \pm \hat y,\pm \hat z\} c.\]
\[ \g^2 = \sin^2\paren{\half{k_xc}} + \sin^2\paren{\half{k_yc}}+\sin^2\paren{\half{k_zc}}.\]
For small magnitudes of $\bv k$, $\g^2$ is isotropic:
\[ \g^2(k \ll \pi/c) \approx \frac{k^2c^2}{4}.\]
And for a triangular lattice:
\[ \{\bv c\} = \{\pm \hat x,\pm \paren{\half{1}\hat x + \half{\sqrt 3} \hat y},\pm \paren{ \half 1 \hat x - \half{\sqrt 3} \hat y}\}c.\]
\[ \g^2 = \sin^2\paren{\half{k_x c}} + \sin^2\paren{\frac{k_x c + \sqrt 3 k_y c}{4}} + \sin^2 \paren{\frac{ k_x c - \sqrt 3 k_y c}{4}}.\]
For $k \ll \pi/c$, we once again see isotropic dispersion:
\[ \g^2 \approx \frac{c^2}{16}\paren{4k_x^2 + k_x^2 + 2\sqrt 3 k_x k_y + 3 k_y^2 + k_x^2 - 2\sqrt 3 k_x k_y + 3 k_y^2} = \frac{3 c^2 k^2}{8}.\]

\section{Vibrational and Librational Coupling}
Now we can apply what we derived above to the problem of vibrational and librational coupling. Librations, from Latin \emph{librare}, to sway  or balance (same as \emph{Libra}, by the way), is an oscillatory mode of an anistropic particle. Vibrations are disturbances in the positional ordering of a lattice of particles, and librations are disturbances in the orientational ordering of a lattice of particles. The goal is to characterize the physics of phonons and \emph{librons}, which is the name of the quasiparticle associated with an excitation of the libration field. Aside: evidently, the moon and Earth exhibit mutual librations, which actually allows us to see more than half the face of the moon. Though the moon is tidally locked with the Earth, it exhibits slow swayings back and forth and different parts of the dark side oscillate into view, permitting us to see about 59\% of the moon's surface. 


Now, the first step is to assume there is a pair-wise potential between the objects that sit at our crystal lattice sites that is dependent not only on the interparticle distance but the interparticle orientations. By translational invariance of the pair of particles, it must be that the potential only depends on the relative positions, but we don't have any reason to suppose that the potential only depends on the relative orientation. However, any orientation dependence can be absorbed into the two rotational orientation variables of our particles, so we need only consider their absolute distance. So we write:
\[ V = V(\abs{\br_1 - \br_2},\th_1,\th_2).\]
Here, I'm putting us in $d=2$, so that we only need one parameter to describe the orientation of our particles. We would like to explore excitations around the \emph{global} ground state. I'm going to suppose that the particle shapes we have are not so terribly complicated such that the unit cell of the crystal has one particle each, and that we can expand the orientations $\th_\br$ around an equilibrium $\th$ that is the same for all particles. It's probably not difficult to come up with a counter-example to this configuration, and I'm sure what would result is higher-frequency optical librons.

I'm going to reparametrize the potential such that its arguments are a bit more intuitive and coincide more closely with what has been done in the group previously. I'll characterize the orientations with a sum $\th_\br + \th_{\br + \bv c} = \th^+_{\bv c}$ and a difference $\th_\br - \th_{\br - \bv c} = \th^-_{\bv c}$, and here, I'm labeling the variables with their lattice sites. Simulations have shown that a displacement in $\th^+$ is significantly easier than a displacement in $\th^-$, which will later give us some spring constants that are very disparate in magnitude to do some nice perturbation theory, I'm sure.

Now, a second order expansion of the potential for only the nearest neighbor coupling yields:
\[ V^{(2)} = \recip{4}\sum_{\bv c} v_i \left.\frac{\del^2 V}{\del v_i \del v_j}\right|_{\br = \bv c} v_j.\]
The $\recip{4}$ is the product of one half from the double counting over lattice spacings, and the other half from the Taylor expansion of the potential. Here, I'm using Einstein summation convention, and the entries of the generalized displacement $v_i$ are:
\[ \bv v = \paren{u_{x,\bv r}-u_{x,\bv r + \bv c},u_{y,\bv r} - u_{y,\bv r + \bv c},\th_\br + \th_{\br + \bv c},\th_\br - \th_{\br + \bv c}}.\]

Thus we have
\[ \frac{\del^2 V}{\del v_i \del v_j} = \pmtrx{\pdert{V}{\D u_x} & \pdder{V}{\D u_x}{\D u_y} & \pdder{V}{\D u_x}{\th^+} & \pdder{V}{\D u_x}{\th^-} \\ \pdder{V}{\D u_y}{\D u_x} & \pdert{V}{\D u_y} & \pdder{V}{\D u_y}{\th^+} & \pdder{V}{\D u_y}{\th^-} \\ \pdder{V}{\th^+}{\D u_x} & \pdder{V}{\th^+}{\D u_y} & \pdert{V}{\th^+} & \pdder{V}{\th^+}{\th^-} \\ \pdder{V}{\th^-}{\D u_x} & \pdder{V}{\th^-}{\D u_y} & \pdder{V}{\th^-}{\th^+} & \pdert{V}{\th^-}}.\]
We have symmetry in the second derivatives, so that constrains this matrix to be symmetric (is this necessarily so in 3D when rotations don't commute with each other?). So we're already down to ten distinct coupling constants. Now we exploit the various symmetry properties of the potential to eliminate some of these independent coupling constants. Let's examine the derivative of the potential with respect to the spatial displacements.
\[ V(x_1-x_2,y_1-y_2,\th_1,\th_2) = V(\sqrt{(x_1-x_2)^2+(y_1-y_2)^2},\th_1,\th_2) = V(r,\th_1,\th_2).\]
\[ \pder{V}{x_1} = \pder{V}{r}\pder{r}{x_1} = \pder{V}{r}\frac{(x_1-x_2)}{r}.\]
\[ \pdder{V}{x_1}{y_1} = -\frac{(x_1-x_2) (y_1-y_2)}{r^3}\pder{V}{r} + \frac{(x_1-x_2)(y_1-y_2)}{r^2} \pdert V r.\]
\[ \pdert{V}{x_1} = \recip{r} \pder V r - \frac{(x_1-x_2)^2}{r^3}\pder V r + \frac{(x_1-x_2)^2}{r^2} \pdert V r.\]
We evaluate these derivatives at the equilibrium position of $V$, which is $r = \abs{c}$, and the first derivatives vanish and we finally end up with:
\[ \pdder{V}{x_1}{y_1} = \frac{c_x c_y}{c^2}\pdert V r = \hat c _x \hat c_y \left. \pdert V r\right |_{\bv c}.\]
\[ \pdert {V }{x_1} = \frac{c_x^2}{c^2} \pdert V r = \hat c_x^2 \left. \pdert V r\right |_{\bv c}.\]
So this is good, we arrive at derivatives involving explicit reference to the crystal lattice vectors, and we extract one common coupling constant: $\pdert V r$. Thus, \emph{ a derivative with respect to a physical displacement is equivalent to a derivative in the radial displacement r and an explicit reference to the crystal lattice vector, which constitutes the equilibrium displacement between two particles.} So let's redefine $\bv v$ to incorporate the crystal lattice unit vectors:
\[ \bv v_{\br,\bv c} = \paren{ (u_{x,\br} - u_{x,\br + \bv c})\hat c_x, (u_{y,\br} - u_{y,\br+\bv c}) \hat c_y, \th_\br + \th_{\br + \bv c},\th_\br - \th_{\br + \bv c}}.\]
And now we only have six coupling constants, which I'll put in a matrix $\bv M$:
\[ \bv M = \pmtrx{\pdert V r & \pdert V r & \pdder{V}{r}{\th^+} &  \pdder{V}{r}{\th^-} \\ \pdert V r & \pdert V r & \pdder{V}{r}{\th^+} & \pdder{V}{r}{\th^-} \\ \pdder{V}{r}{\th^+} & \pdder{V}{r}{\th^+} & \pdert{V}{\th^+} & \pdder{V}{\th^+}{\th^-} \\ \pdder{V}{r}{\th^-} & \pdder{V}{r}{\th^-} & \pdder{V}{\th^+}{\th^-} & \pdert {V}{\th^-}} = \pmtrx{ A & A & B & C \\ A & A & B & C \\ B & B & D & F \\ C & C & F & E}.\]

We can write the kinetic energies:
\[ T = \sum_\br \half{1}m\paren{\dot u_{x,\br}^2 + \dot u_{y,\br}^2} + \recip{2} I \dot \th_\br^2.\]
So now we can write down our Lagrangian:
\[ \mathcal{L} = T - \recip 4 \sum_\br \sum_{\bv c} v_i v_j M_{ij}.\]
At this point, it is probably worth saying that it's far easier to handle the ensuing matrix math if we rescale our generalized coordinates by the square root of their associated mass. That is, we should send
\[ \bv u \goto \frac{\bv u}{\sqrt m},\quad \th \goto \frac{\th}{\sqrt I}.\]
This also scales all the derivatives in the matrix $M$, and we'll just use all the same variable names without fussing around further. Now we write down the equations of motion:
\[  \ddot u_{x,\br} = -\pder{}{u_{x,\br}} \paren{\recip{4}\sum_{\br'}\sum_{\bv c} v_i v_j M_{ij}}.\]
\[ \ddot u_{x,\br} = -\half{1} \sum_{\bv c} \hat c_x  \left [   A\hat c \cdot  (2\bv u_{\br} - \bv u_{\br + \bv c} - \bv u_{\br - \bv c})  + B( \th_{\br + \bv c} - \th_{\br - \bv c}) + C(2\th_\br - \th_{\br +\bv c} - \th_{\br - \bv c})   \right]. \]
\[ \ddot u_{y,\br} = -\half{1} \sum_{\bv c} \hat c_y  \left [   A\hat c \cdot  (2\bv u_{\br} - \bv u_{\br + \bv c} - \bv u_{\br - \bv c})  + B(\th_{\br + \bv c} -\th_{\br - \bv c}) + C(2\th_\br - \th_{\br +\bv c} - \th_{\br - \bv c})   \right]. \]
We can simplify this a bit:
\[ \ddot {\bv u}_{\br} = -\half{1} \sum_{\bv c} \hat c  \left [   A\hat c \cdot  (2\bv u_{\br} - \bv u_{\br + \bv c} - \bv u_{\br - \bv c})  + B(\th_{\br + \bv c} - \th_{\br - \bv c}) + C(2\th_\br - \th_{\br +\bv c} - \th_{\br - \bv c})   \right]. \]
The $\th$ field is a bit trickier, so I'm going to write out the whole thing:
\begin{multline*}
 \ddot \th_{\br} = -\recip{4} \sum_{\bv c} \pder{}{\th_{\br}} \left [ D(\th_\br + \th_{\br + \bv c})^2 + D(\th_{\br - \bv c}+\th_{\br})^2 + E(\th_\br - \th_{\br + \bv c})^2 + E(\th_{\br - \bv c} - \th_{\br})^2 \right .\\ \left.+ 2F(\th_{\br}^2 - \th_{\br + \bv c}^2) + 2F(\th_{\br-\bv c}^2 - \th_\br^2) + 2B \hat c \cdot (\bv u_\br - \bv u_{\br + \bv c}) (\th_\br + \th_{\br + \bv c}) + 2B \hat c \cdot (\bv u_{\br - \bv c} - \bv u_\br) (\th_{\br - \bv c} + \th_{\br})  \right. \\ \left.+ 2C \hat c \cdot (\bv u _\br - \bv u_{\br + \bv c})(\th_\br - \th_{\br + \bv c}) + 2C \hat c \cdot (\bv u_{\br - \bv c} - \bv u_{\br})(\th_{\br - \bv c} - \th_{\br})\right ] .
 \end{multline*}
 We can see right away that the terms involving $F$ actually cancel out upon summation over the crystal lattice sites, so we can just remove them from the coupling matrix. This is a nice unintended consequence of inversion symmetry. Thus, we have
\begin{multline*}   \ddot \th_\br = -\half{1} \sum_{\bv c} \left [ D(2\th_{\br} + \th_{\br + \bv c} + \th_{\br - \bv c}) + E(2\th_{\br} - \th_{\br + \bv c} - \th_{\br - \bv c}) + B \hat c \cdot (- \bv u_{\br + \bv c} + \bv u_{\br - \bv c})  \right. \\ \left.+ C\hat c \cdot (2\bv u_\br - \bv u_{\br + \bv c} - \bv u_{\br - \bv c}) \right ]. \end{multline*}

Now it's time to bring in the Fourier transforms of the fields. Let's define:
\[ \bv u_\br = \sum_{\bv k} \tilde{\bv u}_{\bv k} e^{-i \Dot k r},\quad \th_\br = \sum_{\bv k} \tilde \th_{\bv k} e^{-i \Dot k r}.\]
And just like before, we'll assume time harmonic solutions so we get $-\w^2$ on the left (and I'll drop the dependence on $\bv k$ for the sake of brevity):
\[ \w^2  \tilde{\bv u} = \half{A} \tilde{\bv u} \cdot \sum_{\bv c} \hat c \hat c \paren{2 - e^{-i\Dot k c} - e^{i \Dot k c}} + \half{B} \tilde \th \sum_{\bv c} \hat c\paren{e^{-i \Dot k c} - e^{i \Dot k c}} + \half{C} \tilde \th \sum_{\bv c} \hat c \paren{2 - e^{-i \Dot k c} - e^{i \Dot k c}}.\]
\begin{multline*} \w^2  \tilde \th = \half{D}\tilde \th \sum_{\bv c} \paren{2 + e^{-i \Dot k c} + e^{i \Dot k c}} + \half{E} \tilde \th \sum_{\bv c} \paren{ 2 - e^{- i \Dot k c} - e^{i \Dot k c}} + \half{B} \tilde{\bv u} \cdot \sum_{\bv c} \hat c \paren{e^{i \Dot k c} - e^{-i \Dot k c}} \\ + \half{C} \bv{\tilde u} \sum_{\bv c} \hat c \paren{2 - e^{-i \Dot k c} - e^{i \Dot k c}}. \end{multline*}
Now I'm going to swap out for some trig functions:
\[ \w^2  \tilde{\bv u} = 2A\tilde{\bv u} \cdot \sum_{\bv c} \hat c \hat c \sin^2\paren{\half{\Dot k c}} - i B\tilde \th \sum_{\bv c} \hat c\sinp{\Dot k c} + 2C \tilde \th \sum_{\bv c} \hat c \sin^2\paren{\half{\Dot k c}}.\]
\[ \w^2  \tilde \th = 2D\tilde \th \sum_{\bv c} \cos^2\paren{\half{\Dot k c}}+ 2E \tilde \th \sum_{\bv c} \sin^2\paren{\half{\Dot k c}} + iB \tilde{\bv u} \cdot \sum_{\bv c} \hat c \sinp{\Dot k c} + 2C \tilde{\bv u} \sum_{\bv c} \hat c \sin^2\paren{\half{\Dot k c}}.\]
The sums involving $C$ are odd in $\bv c$, so due to inversion symmetry, they add up to zero. Finally, we have a matricial equation for the phonon and libron modes:
\[ \w^2 \pmtrx{ \tilde u_x \\ \tilde u_y \\ \tilde \th} = \sum_{\bv c} \pmtrx{ 2A \hat c_x^2 \sin^2\paren{\Dot k c/2} & 2A \hat c_x \hat c_y \sin^2 \paren{\Dot k c/2} & - i B \hat c_x \sinp{\Dot k c} \\ 2A \hat c_x \hat c_y \sin^2\paren{\Dot k c/2} & 2A \hat c_y^2 \sin^2 \paren{\Dot k c/2} & - i B \hat c_y \sinp{\Dot k c} \\ i B \hat c_x \sinp{\Dot k c} & i B \hat c_y \sinp{\Dot k c} & 2D \cos^2\paren{\Dot k c/2} + 2E \sin^2 \paren{ \Dot k c/2}}  \pmtrx{ \tilde u_x \\ \tilde u_y \\ \tilde \th}. \]


At this point, it's difficult to move forward in a concrete way without specifying a lattice. As a clear example, let's look at the square lattice:
\[ \{ \bv c \} = \{ \pm c \hat x, \pm c \hat y \}.\]
In this case, the matrix is very simple, and for small $k$, we have:
\[  \w^2 \pmtrx{ \tilde u_x \\ \tilde u_y \\ \tilde \th} = \\ \pmtrx{ Ak_x^2c^2 & 0 & - 2i B k_x c \\ 0 & Ak_y^2 c^2 & -2iBk_y c \\ 2i B k_x c & 2i B k_y c & 8D + Ek^2c^2} \pmtrx{ \tilde u_x \\ \tilde u_y \\ \tilde \th}.  \]

From here, we could diagonalize this matrix equation and find the eigenmodes in terms of the phonons ($\bv{\tilde u}$) and the librons ($\tilde \th$). But I think it's interesting to take the limit as $k\goto 0$. In this limit, the frequency of phonon waves goes to zero, but the frequency of the libration field oscillation is nonzero:
\[ \w^2= 8D.\]
So we find that the libration field is \emph{massive}. Its mass is proportional to $\sqrt{D}$, and $D$ is the elastic constant
\[ D = \recip{I}\pdert{V}{\th^+}.\]
This is the constant that tells us how difficult it is to mutually rotate our squares. Experiment has shown that this elastic constant is far smaller than $D$, the constant associated with rotating our squares antisymmetrically. So in practice, it would probably be difficult to observe this mass.

Furthermore, we see that the elastic coupling matrix has imaginary entries. This tells us that the eigenmodes of the libration and vibration fields are \emph{in general, not in phase with each other.}

For my first project, Wenbo and I are going to be working with the hexagonal lattice formed by hard hexagons in 2D. So we have the set of crystal lattice vectors:
\[ \{ \bv c \} = \{\pm \bv c_1,\pm \bv c_2,\pm \bv c_3\} =    \left \{ \pm c \hat x, \pm \half{c} (\hat x + \sqrt 3 \hat y),\pm \half{c} (\hat x - \sqrt 3 \hat y) \right \}.\]
I've implicitly chosen a particular orientation of the crystal, but this isn't necessary.


\section{Harmonic Hamiltonians and Correlation Functions}
Now we establish some useful results to try to measure the elastic constants defined above in terms of correlation functions in thermodynamics. For this derivation, I will restrict myself to the canonical ensemble. If we are in the $NpT$ ensemble, then I don't think things get much different, but I will point out in the appropriate place in the derivation this difference would need to be accounted for.

Consider the following Hamiltonian:
\[\mathcal{H} = \half{1} \bv p \cdot \bv M ^{-1} \cdot \bv p + \half{1} \bv x \cdot \bv K \cdot \bv x = \half{1}p_i M^{-1}_{ij} p_j + \half{1}x_i K_{ij}x_j.\]
In our case, we'll take $\bv M^{-1}$ to be diagonal, which it almost always is, and $\bv K$ will be a positive definite symmetric matrix, which it always will be for a system with harmonic inter-particle potentials. In our derivation in the last section, the elements of the vector $x_i$ will be all the fields $u_{x,i}$, $u_{y,i}$ and $\th_i$. Then we understand $K_{ij}$ to be an elastic coupling matrix that keeps track of every nearest neighbor interaction. That it is positive definite means that the product $\bv x \cdot \bv K \cdot \bv x$ will always be greater than zero except for $\bv x = 0$, which just tells us that we can't reduce the energy of our system to arbitrarily low energies. So let's first integrate out the momentum degrees of freedom in our partition function. Let's call $m_i$ the $i^\mathrm{th}$ eigenvalue of the mass matrix.
\[ Z = \recip{h^N} \paren{\int \prod_idp_i e^{-\recip{2m_i}\b p_i^2}}\int d\bv x e^{-\recip{2}\b x_iK_{ij}x_j}.\]
\[ Z = \frac{(2\pi k_BT)^{N/2}}{h^N}\sqrt{\prod_i m_i} \paren{\int d\bv x e^{-\recip{2}\b x_iK_{ij}x_j}}.\]
\[ Z = \frac{(2\pi k_BT)^{N/2}}{h^N}\sqrt{\det M} \paren{\int d\bv x e^{-\recip{2}\b x_iK_{ij}x_j}} = Z_0(T,\bv M) \int d\bv x e^{-\recip{2}\b x_iK_{ij}x_j}.\]
To evaluate this Gaussian integral, we diagonalize the matrix $\bv K$ by a unitary transformation into $\bv U^\dagger \bv D \bv U$. We can change variables with impunity, because the transformation is unitary:
\[ \bv n = \bv U \bv x.\]
\[ d\bv n = \abs{\pder{\bv n}{\bv x}} d\bv x.\]
Here, the thing in absolute value bars is the Jacobian matrix of the transformation, which, since the transformation is linear, is simply the matrix $\bv U$ itself. The transformation is unitary so its determinant is 1, and thus $d\bv n = d\bv x$:
\[ \int d\bv x e^{-\half{1}\b x_i K_{ij} x_j} = \int d\bv n e^{-\half{1}\b n_i D_{ij} n_j} = \prod_i \int dn_i e^{-\half{1} \b n_i^2 \lam_i}.\]
Here, $\lam_i$ is the $i^\mathrm{th}$ eigenvalue of $\bv K$ corresponding to the eigenmode $n_i$. Thus we have
\[ \int d\bv x e^{-\half{1}\b x_i K_{ij} x_j} = \prod_i \frac{2\pi}{\lam_i} = \frac{\paren{2\pi}^{N/2}}{\sqrt{\det D}} = \frac{\paren{2\pi}^{N/2}}{\sqrt{\det K}}.\]
It is at \emph{this point} where the ensemble matters. Consider a system of two particles connected by one spring. Their relative positions have an associated potential energy, but their \emph{center of mass position has no energy associated with it}. This is the same as finding that one of the eigenvalues of the $\bv K$ matrix is zero. This isn't a problem, we just have to be careful when doing the Gaussian integral. In this case, we would not integrate over all of space, $-\infty \goto \infty$, but rather we integrate over the volume of the box and we pick up one factor of $V$ which would couple to the external pressure. We can integrate over all of space with the other modes because the exponential factor $e^{-\b k x^2/2}$ dies off sufficiently quickly that we can consider it localized, and we can safely extend the region of integration to all of space.

So now let's look at some correlation functions. We can recast the Hamiltonian in terms of the eigenmodes of the potential, which by their definition are independent degrees of freedom (independent in that their exponential factors in the Boltzmann factor factorize), and so we find:
\[ \avg{n_i} = \recip{Z} \int dn_i \, n_i e^{-\recip 2 \b n_i^2 \lam_i} \prod_{j\neq i} \int dn_j \, e^{-\recip{2}\b n_j^2 \lam_i} = 0.\]
\[ \avg{n_i n_j}_{i\neq j} = \recip{Z} \int dn_i \, n_i e^{-\recip 2 \b n_i^2 \lam_i} \int dn_j \, n_j e^{-\recip 2 n_j^2 \lam _ j} \prod_{k \neq i,j} dn_k e^{-\recip 2 \b n_k^2 \lam_k} = \avg{n_i}\avg{n_j} = 0.\]
\[ \avg{n_i^2} = \recip{Z} \int dn_i \, n_i^2 e^{-\recip 2 \b n_i^2 \lam_i} \prod_{j\neq i} \int dn_j \, e^{-\recip{2}\b n_j^2 \lam_i}.\]
\[ \avg{n_i^2} = -\recip{Z} \pder{Z}{\recip 2 \b \lam_i} = -2k_BT \pder{\log Z}{\lam_i}.\]
We have the partition function explicitly already,
\[ \avg{n_i^2} = -2k_BT \pder{}{\lam_i} \paren{ -\recip{2} \log \det \bv D} = \frac{k_BT}{\lam_i}.\]
So we can summarize:
\[ \avg{n_in_j} = \frac{k_BT}{\lam_i}\d_{ij}.\]
From the first equation regarding $\avg{n_i}$, we have $n_i = U_{ij} x_j$, from which it follows $U^{-1}_{ij} n_j = x_i$ and so
\[ \avg{x_i} = 0.\]
We can in addition compute all the correlation functions:
\[ \avg{x_i x_j} = \avg{U^{-1}_{ik}n_k U^{-1}_{j\ell}n_\ell} = U^\dagger_{ik} U_{\ell j} \paren{\frac{k_BT}{\lam_k}}\d_{k\ell}.\]
But $\lam_k$ are the elements of the diagonal matrix $D_{k\ell} = \lam_k \d_{k\ell}$, so
\[ \avg{x_i x_j} = k_BT U^\dagger_{ik} D^{-1}_{k\ell} U_{\ell j}.\]
\[ \boxed{\avg{x_i x_j} = k_BT K^{-1}_{ij}.}\]
We could have done this in the $x$ basis without switching to the $n$ basis, but this is much clearer to me.

The power of this formulation is that we now do not need to characterize our lattice in terms of Fourier modes in order to correlate the position and orientation of one particle with the position and orientation of another. However, I would guess that doing a Fourier transform is a computationally easier thing to do than to compute the inverse of a $dN \times dN$ matrix. In fact, by doing the appropriate unitary transformation, we can compute the correlation function of \emph{any linear combination of the field values} and get an entry of the inverse coupling constant matrix.

Side cautionary note: it's entirely possible that $\bv K$ has no inverse because it has one or more zero eigenvalues. As I alluded to before, the zero-eigenvalue eigenmodes can then explore all of their configuration space and are not constrained by being Gaussian distributed. In this case, we must single out these free modes and find their mean value and variance, then make a new matrix $\bv D'$ excluding these floppy degrees of freedom.

\section{Measurements \emph{in silico}}

I generally try not to miss opportunities to use goofy Latin phrases. Anyway, all of the previous derivations are cool, but we now need some way to efficiently measure these elastic constants from the correlations functions. Of particular importance is $G$, the elastic coupling constant that shows up in the equations of motion for the Fourier modes on page 7.

A particularly useful basis to work in is the Fourier basis: though the fields are still coupled to each other, their Fourier components \emph{are decoupled with respect to wavelength}. That is, a phonon of wavelength $\bv k_1$ may interact with librons of wavelength $\bv k_1$, but it will not interact with phonons or librons of wavelength $\bv k_2$. The matrix equation of motion for the Fourier modes given on page 7 guarantees that modes of different wavelengths are uncoupled.

Furthermore, the matrix of coupling constants in real space is sparse, but it has many off-diagonal components. Schematically speaking:
\[ \bv K_{\br} = \pmtrx{ k_1 & - k_{12} & 0 & 0 & 0 & \ldots \\ -k_{12} & k_2 & -k_{23} & 0 & 0 & \\ 0 & -k_{23} & k_3 & -k_{34} & 0 & \\ 0 & 0 & -k_{34} & k_4 & -k_{45} & \\ 0 & 0 & 0 & -k_{45} & k_5 & \\ \vdots & & & & & \ddots }.\]
This guarantees that the inverse matrix $\bv K_\br^{-1}$ is \emph{the exact opposite of sparse, i.e.} every element is nonzero (if the inverse exists). On the other hand, in the Fourier basis, because modes of different wavelength are uncoupled, the matrix has a very useful structure. If we let:
\[ \bv E(\bv k) = 4\pmtrx{A\g^2 & 0 & G\g^2 \\ 0 & A\g^2 & -G\g^2 \\ G \g^2 & - G\g^2 & C\mu^2 + D\g^2}\pmtrx{\tilde u_x \\ \tilde u_y \\ \tilde \th},\]
as I write on page 7, with $\g$ and $\mu$ defined previously, then we can write the total coupling matrix in a block diagonal form:
\[ \bv K_{\bv k} = \pmtrx{\bv E(\bv k_0) & 0 & 0 & \ldots \\ 0 & \bv E(\bv k_1) & 0 & \\ 0 & 0 & \bv E(\bv k_2) & \\ \vdots & & & \ddots}.\]
The inverse is simply
\[ \bv K^{-1}_{\bv k} = \pmtrx{\bv E^{-1}(\bv k_0) & 0 & 0 & \ldots \\ 0 & \bv E^{-1}(\bv k_1) & 0 & \\ 0 & 0 & \bv E^{-1}(\bv k_2) & \\ \vdots & & & \ddots},\]
so this inverse matrix is also sparse. The inverse of $\bv E$ is (\emph{grace \`{a}} Mathematica):
\[\bv E^{-1}(\bv k) =  \pmtrx{(AD-G^2)\g^2 + A C\mu^2 & -G^2 \g^2 & -A G \g^2 \\ -G^2 \g^2 & (AD-G^2)\g^2 + AC \mu^2 & AG\g^2 \\ -A G\g^2 & AG \g^2 & A^2 \g^2}\recip{4A\g^2\paren{(AD-2G^2)\g^2 + A C \mu^2}}.\]
Now, it's entirely possible that the determinant is zero, depending on the constants $A,C,D,$ and $G$. Also, this matrix clearly does not exist for $\g=k = 0$. This is because phonon modes with $\bv k = 0$ correspond to uniform displacements of the lattice, which are modes I foretold earlier which contribute a factor of $V$ to the partition function. So we wouldn't want to look at the $\bv k  =  0$ modes to help us figure out these coupling constants. But this doesn't prevent us from asking about correlations within modes of $\bv k \neq 0$. So if I am able to compute $\avg{\tilde \th_\bv k \tilde u_{x,\bv k}}$, then that gives me the numerical value of
\[ \avg{\tilde \th_{\bv k} \tilde u_{x,\bv k}} = \frac{-G}{4\paren{(AD-2G^2)\g^2 + AC\mu^2}}k_BT.\]
We can compute the three other independent correlation functions and solve a system of equations to get all the coupling constants!



%%%%%%%%%
\end{document} 

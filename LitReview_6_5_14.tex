\documentclass[11pt]{amsart}
\usepackage{geometry}                % See geometry.pdf to learn the layout options. There are lots.
\geometry{letterpaper}                   % ... or a4paper or a5paper or ... 
%\geometry{landscape}                % Activate for for rotated page geometry
%\usepackage[parfill]{parskip}    % Activate to begin paragraphs with an empty line rather than an indent
\usepackage{graphicx}
\usepackage{amssymb}
\usepackage{epstopdf}
\DeclareGraphicsRule{.tif}{png}{.png}{`convert #1 `dirname #1`/`basename #1 .tif`.png}

\title{Lit Review 6/5/14}
\author{Jim Antonaglia}
\date{}                                           % Activate to display a given date or no date

\begin{document}
\maketitle
\section{Magnetic Wood}
Authors cooked various iron solutions into the wood, and the iron formed magnetic crystals within the cell walls. It wasn't expected that the irons would pick a preferred direction, but they preferred to align along the "longitudinal" direction, i.e. up and down the trunk of the tree. The resulting magnetic wood was mostly paramagnetic, and it lined up and flipped around in response to an external field. People are now proposing to use magnetic wood as an electromagnetic absorber.

\section{Machine Learning Algorithm Repository}
Machine learning algorithms need training sets and running learning experiments with various learning algorithms. The repo will store "predictions for each test instance as well as the learning algorithm, hyper parameters, and training sets," which will help facilitate metalearning: data on learning algorithms. There already exist databases like this, but this one focuses on ease of access, direct downloadability of all sorts of data, open access, and query-ability.

\emph{Machine Learning Results Repository} (MLRR) provides data sets ready for download for meta-learning problems.

\section{Superelastic Organic Crystals}
Superelasticity is a misnomer: it should be referred to as pseudoelasticity. Upon significant stress, the crystal structure changes from one phase to another, and when the stress is relieved, the crystal assumes its original crystal phase, which gives rise to its other name: shape-memory alloys. They've always been in the context of metal alloys, but not an organic system.

Terephthalamide (benzene ring with some amines on the ends), makes two crystal phases. The $\alpha$ phase is a polymeric sheet that's repeating with one molecule per unit cell. In $\beta$ phase, the unit cell is more complex and the benzene rings are all twisted in various ways, but this is a more dense packing I guess. They got it to store elastic energy with $92.5\%$ efficiency with shears up to $11.3\%$.

\section{Self-reproducing Inorganic Colloidosomes}
Instead of lipid membranes, the membrane is an ultrathin layer of silica nano particles surrounding stable droplets of water. They first make the water droplets with Pickering emulsion, then bind the silica particles with oligomeric crosslinkings. The binding produces methanol, which doesn't like to be in the oil on the exterior, and so it diffuses into the cell, and the cell grows in size and burgeons with pressure. At a particular critical pressure, the water/methanol leaks out and buds into a new cell, which gathers around it some more silica nanoparticles until the cell totally buds and seals itself.

\section{Open Structures with Patchy Colloidal Particles}
You got some triblock Janus particles which can form, depending on their patch sizes, a few different kinds of open lattices. Open lattices exhibit things like a negative Poisson ratio, negative thermal expansion, and "holographic elasticity," meaning that information about the bulk is encoded on the surface, where the external constraints are.

They find that the bonds are stabilized by an effective bending and twisting modulus at each lattice site which comes from entropic considerations: the more ways the spheres can pivot around in their sockets, the better the bond. With no bending modulus, they have lots and lots of zero-frequency (floppy) modes, but when you include the bending modulus, you lift the floppy modes and they become rigid.

Also, look at this phase diagram. Have you seen a more MSPaint phase diagram?

\section{Orientationally Glassy Crystals of Janus Spheres}
Repulsive edges are electrostatically charged, and for small salt concentrations, the electrostatic interactions dominate, and in 2D, they crystallize into hexagonal packing with orientational disorder (orientational liquid). With added salt, the electrostatic interactions get screened, and the orientation begins to matter more, and we develop stripes. Dynamically speaking, the relaxation time increases exponentially as salt is added. Dynamics are shown in a movie: we have glassy dynamics, pretty cool.

\section{Cancer and Entropy}
They denote measurable quantities with $x$ and immeasurable (internal) quantities with $q$, and $i$ is the population index. They qualify the entropy of an infected patient and a non-infected patient using some a priori known samples. Then they have a baseline for the entropy of distributions of given qualities which they then use to compare to a priori undiagnosed samples of cells.

\section{Quasicrystal Growth Modes}
They start with a \emph{dynamical} phase-field crystal model and show that they need a particular size of seed to get a quasicrystal from a supercooled fluid. They find two different growth regimes, for different parameter ranges. For parameters near the triple point of 12-fold QC, triangular phase, and fluid phase, the growth of the quasicrystal is more or less without defects and with a broad propagation front. For parameters very far from the triple point, they find that the quasicrystal is sensitive to the initial properties of the seed, the propagation front is sticky and goes to fast and goes back to sticky, and there are more defects.

In this second mode of growth, the dislocations are phasonic excitations.

Two different growth modes come from different interactions of length scales. Near the triple point, both length scales are important, whereas far from the triple point, one scale dominates the other. The authors suggest that quasicrystals with more than two incommensurate length scales will be difficult to grow.

\section{Particle Selection in Nematic Colloids}
When we impose homeotropic (perpendicular) anchoring of the surrounding nematic liquid at the surface of a plane, we can put a dimple in the wall and spherical colloids will collect at those dimples rather than be repelled from the walls (assuming the LC aligns homeotropically with the spherical colloid as well). We can change the surface patterns (using photosensitive azobenzene attached to the colloids themselves) to switch the conditions and sort the particles, because we have some gathering at concave dimples and some gathering at convex dimples.

\section{Active Swimmers and Turbulence}
They had some active particles with a particular interaction potential. They could generate a state of ordered vortices and a state of polar order (large bands?) They find at the crossover, they get \emph{mesoscale turbulence}.




\end{document}  

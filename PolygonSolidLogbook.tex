\documentclass[12pt]{article}
\usepackage[top = 3cm,bottom=3cm,left=2cm,right=2cm]{geometry}        
\geometry{letterpaper}
\usepackage[parfill]{parskip}  
\usepackage{graphicx}
\usepackage{amssymb}
\usepackage{epstopdf}
\usepackage{listings}
\usepackage{color}
\usepackage{amsmath}
\usepackage{tikz}
\usepackage{mathtools}
\usetikzlibrary{decorations.markings,arrows}
\usepackage{bm}
\usepackage{subcaption}
 
 \begin{document}  
 
 \tableofcontents
 
\baselineskip 2.7ex
\parskip 3.5ex

\pagestyle{myheadings}
\markright{James Antonaglia \hfill Polygon Crystal Logbook \hfill}


%Equations/Entries
\newcommand{\beq}{\begin{equation}}
\newcommand{\bequo}{\begin{quotation}}
\newcommand{\beqa}{\begin{eqnarray}}
\newcommand{\eeq}{\end{equation}}
\newcommand{\leeq}[1]{\label{#1}\end{equation}}
\newcommand{\equo}{\end{quotation}}
\newcommand{\eeqa}{\end{eqnarray}}
\newcommand{\non}{\nonumber}
\newcommand{\mx}{\mbox}
\newcommand{\mxf}[1]{\mbox{\footnotesize{#1}}}
\newcommand{\lb}{\label}
\newcommand{\fr}[1]{(\ref{#1})}
\newtheorem{entry}{}[section]
\newcommand{\bent}[1]{\vspace*{-2cm}\hspace*{-1cm}\begin{entry}\lb{e{#1}}\rm}
\newcommand{\eent}{\end{entry}}
\newcommand{\fre}[1]{{\bf\ref{e{#1}}}}
\newcommand{\Emark}{$\sqcap\hspace{-2.7mm}\sqcup$}
\newcommand{\sEmark}{{\fns $\sqcap\hspace{-2.3mm}\sqcup$}}
\newcommand{\fn}{\footnote}


%Greek Letters
\renewcommand{\a}{\alpha}
\renewcommand{\b}{\beta}
\newcommand{\g}{\gamma}
\newcommand{\G}{\Gamma}
\renewcommand{\d}{\delta}
\renewcommand{\th}{\theta}
\renewcommand{\k}{\kappa}
\newcommand{\Th}{\Theta}
\newcommand{\D}{\Delta}
\newcommand{\e}{\epsilon}
\newcommand{\ep}{\varepsilon}
\newcommand{\s}{\sigma}
\renewcommand{\S}{\Sigma}
\newcommand{\w}{\omega}
\newcommand{\W}{\Omega}
\newcommand{\al}{\alpha}
\newcommand{\bet}{\beta}
\newcommand{\gam}{\gamma}
\newcommand{\lam}{\lambda}
\newcommand{\Lam}{\Lambda}
\newcommand{\eps}{\varepsilon}
\newcommand{\ichi}\sichi
\renewcommand{\ni}{\sni}
\renewcommand{\r}{\rho}
\renewcommand{\t}{\tau}
\newcommand{\ph}{\varphi}
\newcommand{\sichi}{{\mbox{{\footnotesize I}}}}
\newcommand{\sni}{{\mbox{{\footnotesize II}}}}

%color2010/6/9
\newcommand{\red}{\color{red}}
\newcommand{\blue}{\color{blue}}
\newcommand{\green}{\color{green}}
\definecolor{gray}{rgb}{0.5, 0.5, 0.5}
\newcommand{\gray}{\color{gray}}

%Derivatives
\newcommand{\pder}[2]{\frac{\partial {#1}}{\partial {#2}}}
\newcommand{\pdert}[2]{\frac{\partial^2 {#1}}{\partial {#2}^2}}
\newcommand{\fder}[2]{\frac{\delta {#1}}{\delta {#2}}}
\newcommand{\PDD}[3]{\left.\frac{\partial^{2}{#1}}{\partial{#2}^{2}}\right|_{#3}
}
\newcommand{\PD}[3]{\left.\frac{\partial{#1}}{\partial{#2}}\right|_{#3}}
\newcommand{\der}[2]{\frac{d {#1}}{d {#2}}}
\newcommand{\pdder}[3]{\frac{\del^2 {#1}}{\del {#2} \del {#3}}}


\renewcommand{\deg}{^\circ}
\newcommand{\com}{{\bf [C] }}
\newcommand{\cend}{\Emark\[\]\vspace*{-1. cm}}
\newcommand{\x}{\times}

%My commands
\newcommand{\win}{\ddot\smile}
\newcommand{\lose}{\ddot\frown}
\newcommand{\avg}[1]{\left \langle #1 \right \rangle}
\newcommand{\E}[1]{\ensuremath{\times10^{#1}}}
\newcommand{\abs}[1]{\ensuremath{\left | #1 \right |}}
\newcommand{\paren}[1]{\left(#1\right)}
\newcommand{\recip}[1]{\frac{1}{#1}}
\newcommand{\ex}[1]{\mathbb{E}[#1]}
\newcommand{\bprob}[1]{\textbf{#1~---}}
\newcommand{\unitv}[1]{\ensuremath{\mathbf{\hat{e}}_{#1}}}
\newcommand{\goto}{\rightarrow}
\newcommand{\expct}[1]{\mathbb{E}[#1]}
\newcommand{\mtrx}[1]{\begin{matrix}#1\end{matrix}}
\newcommand{\pmtrx}[1]{\paren{\begin{matrix}#1\end{matrix}}}
\newcommand{\cosp}[1]{\cos{\paren{#1}}}
\newcommand{\sinp}[1]{\sin{\paren{#1}}}
\newcommand{\tanp}[1]{\tan{\paren{#1}}}
\newcommand{\half}[1]{\frac{#1}{2}}
\newcommand{\ham}{\mathcal{H}}
\newcommand{\tr}{\mathrm{Tr}}
\newcommand{\bv}[1]{\mathbf{#1}}
\newcommand{\Der}[2]{\frac{d#1}{d#2}}
\renewcommand{\Dot}[2]{\ensuremath{\bv{#1}\cdot\bv{#2}}}
\newcommand{\Cross}[2]{\ensuremath{\bv{#1}\times\bv{#2}}}
\newcommand{\del}{\ensuremath{\partial}}
\newcommand{\R}{\ensuremath{\bv{r-r'}}}
\newcommand{\aR}{\ensuremath{\abs{\R}}}
\newcommand{\br}{\ensuremath{\bv{r}}}
\newcommand{\impl}{\ensuremath{\quad \Rightarrow \quad}}
\renewcommand{\div}[1]{\nabla \cdot \bv{#1}}
\newcommand{\curl}[1]{\nabla \times \bv{#1}}
\newcommand{\lapl}{\nabla^2}
\newcommand{\vint}{\int d^3r}
\newcommand{\oocs}{\recip{c^2}}
\newcommand{\mnfp}[1]{\frac{\mu_0 #1}{4\pi}}
\renewcommand{\iiint}{\int_{-\infty}^{\infty}}
\newcommand{\tpi}[1]{\paren{2\pi}^{#1}}
\newcommand{\ootpi}[1]{\recip{\paren{2\pi}^{#1}}}

%%%%%%%%%%%%%%%%%%%%%%%%%%%%%%%

\section{Overview}
This document is meant to log my progress in the investigation of mechanical 
modes in crystals of hard regular polygon particles. The system consists of $N 
= n^2\times n_{\mathrm{unit}}$ particles, where $n$ is some integer and 
$n_\mathrm{unit}$ is the number of particles in the unit cell. The system is 
initialized in its densest known packing and then the crystal vectors are 
enlarged by constant factors to achieve a prescribed packing fraction. The 
motion of particles is decomposed into $n\times n$ distinct Fourier modes. 
There are $3n_\mathrm{unit}$ relevant fields to investigate. The first 
$2n_\mathrm{unit}$ are phonon fields signified by $\bv{\tilde u} = 
\paren{\tilde u_x, \tilde u_y}$ (if the number of particles per unit cell is 
greater than 1, then these fields are further indexed \emph{i.e.\!} $\bv{\tilde 
u}^{\a,\b,\g\ldots}$. The remaining $n_\mathrm{unit}$ fields are libron fields, 
signified by $\tilde \th^{\a,\b,\g\ldots}$. In the case of hexagons, 
$n_\mathrm{unit} = 1$, and in the case of many other polygons including 
pentagons studied here, $n_\mathrm{unit} = 2$.

The various control parameters in a simulation are $n$ the number of unit 
cells along one side of the box, $\phi$ the packing fraction, $s$ the number of 
simulation steps, and the polygon shape. The densest packing, and thus the 
crystal lattice vectors, are determined by the polygon. The real space lattice 
vectors given by $\bv a_1, \bv a_2$ are usually meant to be in the densest 
packing, $\phi_\mathrm{max}$. When the system is relaxed to a smaller packing 
fraction, the lattice vectors are presumed to be $\bv a_i(\phi) = \bv a_i 
\sqrt{\phi_\mathrm{max}/\phi}$. The reciprocal lattice vectors $\bv G_i$ are 
constructed from these real space lattice vectors, and the unit cell of the 
reciprocal lattice forms the Brillouin zone within which $\bv k$ vectors are 
sampled. The relevant $\bv k$ vectors to sample are $n_1 \bv G_1/n + n_2 \bv 
G_2/n$, where $n_i$ are integers. This way, the same number of Fourier modes 
are sampled as there are degrees of freedom in the real space system.

The Fourier modes are analyzed by considering their correlation functions. In 
Monte Carlo, dynamical information is not available, so collective degrees of 
freedom need to be analyzed from correlation functions by way of a kind of 
generalized law of equipartition. We assume that this hard particle system 
which is controlled exclusively by entropic interactions can be approximated by 
a harmonic Hamiltonian:
\beq \mathcal{H} = \recip{2m} p_i \d_{ij} p_j + \recip{2I} 
L_i \d_{ij} L_j + \half 1 x_i K_{ij} x_j. \eeq
The first term is the translational kinetic energy and the second term is the 
rotational kinetic energy. The last term is a harmonic coupling by a $3N\times 
3N$ matrix, and $x_i$ represent both the particle positions and particle 
orientations. In thermal equilibrium, it is provable that
\beq \avg{x_i x_j} =  k_BTK_{ij}^{-1}. \eeq
The matrix $K_{ij}$ is partially diagonalized by the Fourier transform with 
respect to the lattice vectors $\bv a_{1,2}(\phi)$, so we find that
\beq \avg{\tilde x^*_{\mu,\bv k} \tilde x_{\nu, \bv k} } = C_{\mu\nu,\bv k} 
= k_BT K_{\mu\nu,\bv k}^{-1}. \eeq
This decomposes the large $3N\times 3N$ matrix $K_{ij}$ into a block diagonal 
$n^2\times n^2$ matrix whose diagonal elements are $3n_\mathrm{unit} \times 
3n_\mathrm{unit}$ matrices $K_{\mu\nu}$. The modes $\tilde x_\mu$ can represent 
either $\bv{\tilde u}_x^{\a,\b\ldots}$, $\bv{\tilde u}_y^{\a,\b\ldots}$ or 
$\tilde \th^{\a,\b\ldots}$. The ultimate output of the analysis is the matrix 
$K_{\mu\nu,\bv k}$, which is obtained by computing every correlation 
function of all modes of the same wave vector and inverting the resulting matrix 
$C_{\mu\nu,\bv k}$. The elements of $\bv K_{\bv k}$ are the squares of the 
phonon and libron dispersion relations, \emph{i.e.\!} they furnish $\w^2(\bv 
k)$ for all $\bv k$ sampled in the Brillouin zone and for all collective modes 
$\bv {\tilde u}^{\a,\b\ldots}$ and $\tilde \th^{\a,\b\ldots}$.

The matrices $\bv K_{\bv k}$ depend on the packing 
fraction $\phi$, the number of particles $N$, and the number of Monte 
Carlo sweeps taken $s$. If $s$ is too small, then there is not a sufficient 
number of statistically independent Monte Carlo frames to compute the elements 
of the correlation matrix $\bv C$. The effect of $s$ will be explored in Section 
2.

In principle, the matrix $\bv C$ should be independent of $N$ (or equivalently 
$n$), but having larger $N$ helps resolve $\bv C$ as a function of $\bv k$, the 
wavevector, as the Brillouin zone is more densely populated. The effect of $N$ 
will be explored in Section 3.

The matrix $\bv C$ is most markedly affected by changes in the packing 
fraction. Because some polygon crystals melt into various phases (hexatic, 
rotator, or plain fluid), particular elements of $\bv C$ should simply vanish 
across these transitions, which I will show leads to flat dispersion relations. 
That is, particular elements of $\bv K_{\bv k}$ become independent of $\bv k$. 
The effect of packing fraction will be shown in Section 4.

One of the difficulties in discerning meaningful interpretations of $\bv K_{\bv 
k}$ is obviously the high dimensionality of the matrix. It is necessarily 
symmetric, so it has $3n_\mathrm{unit} (3n_\mathrm{unit}+1)/2$ independent 
components, each of which is a function of the two-dimensional $\bv k$. Even 
for the one-particle unit cell of the hexagon crystal, there are 6 independent 
elements of $\bv K_{\bv k}$ that need to be explored. Luckily, at quadratic 
order, the phonon mode is uncoupled from the libron mode, so this reduces the 
6 components to 4 components (this uncoupling will also be demonstrated 
explicitly in Section 5).


%%Finite size effects


%%Timescale effects





%%Packing fraction effects





%%Nonlinear effects









\end{document}
\documentclass[12pt]{article}
\usepackage[top = 3cm,bottom=3cm,left=2cm,right=2cm]{geometry}        
\geometry{letterpaper}
\usepackage[parfill]{parskip}  
\usepackage{graphicx}
\usepackage{amssymb}
\usepackage{epstopdf}
\usepackage{listings}
\usepackage{color}
\usepackage{amsmath}
\usepackage{tikz}
\usepackage{mathtools}
\usetikzlibrary{decorations.markings,arrows}
\usepackage{bm}
 
 \begin{document}  
 
\baselineskip 2.7ex
\parskip 3.5ex

\pagestyle{myheadings}
\markright{James Antonaglia \hfill Phonon Libron Spectra \hfill}


%Equations/Entries
\newcommand{\beq}{\begin{equation}}
\newcommand{\bequo}{\begin{quotation}}
\newcommand{\beqa}{\begin{eqnarray}}
\newcommand{\eeq}{\end{equation}}
\newcommand{\leeq}[1]{\label{#1}\end{equation}}
\newcommand{\equo}{\end{quotation}}
\newcommand{\eeqa}{\end{eqnarray}}
\newcommand{\non}{\nonumber}
\newcommand{\mx}{\mbox}
\newcommand{\mxf}[1]{\mbox{\footnotesize{#1}}}
\newcommand{\lb}{\label}
\newcommand{\fr}[1]{(\ref{#1})}
\newtheorem{entry}{}[section]
\newcommand{\bent}[1]{\vspace*{-2cm}\hspace*{-1cm}\begin{entry}\lb{e{#1}}\rm}
\newcommand{\eent}{\end{entry}}
\newcommand{\fre}[1]{{\bf\ref{e{#1}}}}
\newcommand{\Emark}{$\sqcap\hspace{-2.7mm}\sqcup$}
\newcommand{\sEmark}{{\fns $\sqcap\hspace{-2.3mm}\sqcup$}}
\newcommand{\fn}{\footnote}


%Greek Letters
\renewcommand{\a}{\alpha}
\renewcommand{\b}{\beta}
\newcommand{\g}{\gamma}
\newcommand{\G}{\Gamma}
\renewcommand{\d}{\delta}
\renewcommand{\th}{\theta}
\renewcommand{\k}{\kappa}
\newcommand{\Th}{\Theta}
\newcommand{\D}{\Delta}
\newcommand{\e}{\epsilon}
\newcommand{\ep}{\varepsilon}
\newcommand{\s}{\sigma}
\renewcommand{\S}{\Sigma}
\newcommand{\w}{\omega}
\newcommand{\W}{\Omega}
\newcommand{\al}{\alpha}
\newcommand{\bet}{\beta}
\newcommand{\gam}{\gamma}
\newcommand{\lam}{\lambda}
\newcommand{\Lam}{\Lambda}
\newcommand{\eps}{\varepsilon}
\newcommand{\ichi}\sichi
\renewcommand{\ni}{\sni}
\renewcommand{\r}{\rho}
\renewcommand{\t}{\tau}
\newcommand{\ph}{\varphi}
\newcommand{\sichi}{{\mbox{{\footnotesize I}}}}
\newcommand{\sni}{{\mbox{{\footnotesize II}}}}

%color2010/6/9
\newcommand{\red}{\color{red}}
\newcommand{\blue}{\color{blue}}
\newcommand{\green}{\color{green}}
\definecolor{gray}{rgb}{0.5, 0.5, 0.5}
\newcommand{\gray}{\color{gray}}

%Derivatives
\newcommand{\pder}[2]{\frac{\partial {#1}}{\partial {#2}}}
\newcommand{\pdert}[2]{\frac{\partial^2 {#1}}{\partial {#2}^2}}
\newcommand{\fder}[2]{\frac{\delta {#1}}{\delta {#2}}}
\newcommand{\PDD}[3]{\left.\frac{\partial^{2}{#1}}{\partial{#2}^{2}}\right|_{#3}
}
\newcommand{\PD}[3]{\left.\frac{\partial{#1}}{\partial{#2}}\right|_{#3}}
\newcommand{\der}[2]{\frac{d {#1}}{d {#2}}}

\renewcommand{\deg}{^\circ}
\newcommand{\com}{{\bf [C] }}
\newcommand{\cend}{\Emark\[\]\vspace*{-1. cm}}
\newcommand{\x}{\times}

%My commands
\newcommand{\win}{\ddot\smile}
\newcommand{\lose}{\ddot\frown}
\newcommand{\avg}[1]{\left \langle #1 \right \rangle}
\newcommand{\E}[1]{\ensuremath{\times10^{#1}}}
\newcommand{\abs}[1]{\ensuremath{\left | #1 \right |}}
\newcommand{\paren}[1]{\left(#1\right)}
\newcommand{\recip}[1]{\frac{1}{#1}}
\newcommand{\ex}[1]{\mathbb{E}[#1]}
\newcommand{\bprob}[1]{\textbf{#1~---}}
\newcommand{\unitv}[1]{\ensuremath{\mathbf{\hat{e}}_{#1}}}
\newcommand{\goto}{\rightarrow}
\newcommand{\expct}[1]{\mathbb{E}[#1]}
\newcommand{\mtrx}[1]{\begin{matrix}#1\end{matrix}}
\newcommand{\pmtrx}[1]{\paren{\begin{matrix}#1\end{matrix}}}
\newcommand{\cosp}[1]{\cos{\paren{#1}}}
\newcommand{\sinp}[1]{\sin{\paren{#1}}}
\newcommand{\tanp}[1]{\tan{\paren{#1}}}
\newcommand{\half}[1]{\frac{#1}{2}}
\newcommand{\ham}{\mathcal{H}}
\newcommand{\tr}{\mathrm{Tr}}
\newcommand{\bv}[1]{\mathbf{#1}}
\newcommand{\Der}[2]{\frac{d#1}{d#2}}
\renewcommand{\Dot}[2]{\ensuremath{\bv{#1}\cdot\bv{#2}}}
\newcommand{\Cross}[2]{\ensuremath{\bv{#1}\times\bv{#2}}}
\newcommand{\del}{\ensuremath{\partial}}
\newcommand{\R}{\ensuremath{\bv{r-r'}}}
\newcommand{\aR}{\ensuremath{\abs{\R}}}
\newcommand{\br}{\ensuremath{\bv{r}}}
\newcommand{\impl}{\ensuremath{\quad \Rightarrow \quad}}
\renewcommand{\div}[1]{\nabla \cdot \bv{#1}}
\newcommand{\curl}[1]{\nabla \times \bv{#1}}
\newcommand{\lapl}{\nabla^2}
\newcommand{\vint}{\int d^3r}
\newcommand{\oocs}{\recip{c^2}}
\newcommand{\mnfp}[1]{\frac{\mu_0 #1}{4\pi}}
\renewcommand{\iiint}{\int_{-\infty}^{\infty}}
\newcommand{\tpi}[1]{\paren{2\pi}^{#1}}
\newcommand{\ootpi}[1]{\recip{\paren{2\pi}^{#1}}}

%%%%%%%%

\section{Coupled Lattice Fields}
I'm going to write a simple toy model to illustrate the approach to solving a system of equations to find the dispersion relation for some coupled fields on our lattice.
\subsection{1D Chain}
Let's start simple, with a one dimensional chain with some harmonic coupling between some arbitrary fields $a_i$ and $b_i$ that are defined on our one dimensional lattice of lattice spacing $\ell$. The potential energy will be
\[ V = \sum_i \half{1}A (a_i-a_{i+1})^2 + \half{1} B(b_i - b_{i+1})^2 + C(a_i-a_{i+1})(b_i - b_{i+1}) + \half{1} M a_i^2.\]
The first two terms are the coupling between nearest neighboring lattice sites, the third term couples the $a$ fields to the $b$ fields, and the last term we'll see is what will give the $a$ modes a mass, \emph{i.e.} their frequencies don't go to zero at zero wavelength. Let's give the $a$ particles a mass $m$ and the $b$ particles a mass $\mu$. So the Lagrangian looks like
\[ \mathcal{L} = \sum_i \half{1}m \dot{a}_i^2 + \half{1}\mu \dot{b}_i^2 - V.\]
Newton's equations then read:
\[ m\ddot{a}_i = -2Aa_i + Aa_{i+1} +Aa_{i-1} - 2Cb_i +Cb_{i+1} + Cb_{i-1} - Ma_i.\]
\[ m\ddot{b}_i = -2Ca_i + Ca_{i+1} +Ca_{i-1} - 2Bb_i +Bb_{i+1} + Bb_{i-1}.\]
Now we go to Fourier space to turn this difference equation into an algebraic equation. The transformation is:
\[ a_j = \sum_j \tilde{a}(k) e^{-i k(j\ell)},\qquad b_j = \sum_j \tilde{b}(k) e^{-ik(j\ell)}.\]
We put in these transforms in the above difference equations and exploit the orthonormality of the exponential to get equality term by term:
\[ m\ddot{\tilde{a}} = -2A\tilde{a} + A\tilde{a}e^{-ik\ell} + A\tilde{a}e^{ik\ell} - 2C \tilde{b} + C \tilde b e^{-ik\ell} + C\tilde b e^{i k \ell} - M\tilde a.\]
\[ \mu\ddot{\tilde{b}} = -2C\tilde{a} + C\tilde{a}e^{-ik\ell} + C\tilde{a}e^{ik\ell} - 2B \tilde{b} + B \tilde b e^{-ik\ell} + B\tilde b e^{i k \ell}.\]
We then look for time-harmonic solutions such that $\der{^2}{t^2}\goto -\w^2$, and we finally have a matrix equation, setting $\sinp{k\ell/2} = \g$:
\[ \pmtrx{m\w^2 - 4\g^2 A - M & -4\g^2 C \\ - 4\g^2 C & \mu \w^2 - 4\g^2 B} \pmtrx{\tilde a \\ \tilde b} = 0.\]
Now if we wanted to, we could put this in Mathematica and get the eigenmodes and corresponding dispersion relations. But let's just set $k=0$ so $\gamma = 0$:
\[ \pmtrx{m\w^2 - M & 0 \\ 0 & \mu \w^2}\pmtrx{\tilde a \\ \tilde b} = 0.\]
The solution to this eigenvalue problem is $\w^2 = 0$ and $\w^2 = M/m$. The first case has eigenvector $\tilde a = 0$ and $\tilde b = b_0$, so this is a wave with zero frequency and zero wavelength with arbitrary amplitude, which corresponds to a global constant shift of the $b$ field. This doesn't couple to the $a$ field. However, for the other eigenvalue, $\w = \sqrt{M/m}$ is a finite frequency with $k=0$ and $\tilde a = a_0$ and $\tilde b = 0$. This is a global and spatially homogenous oscillation of the $a$ field.

\subsection{2D Spring Field}
Now let's consider two coupled fields in two dimensions situated on a lattice. Suppose that we don't need a basis to characterize our unit cell (if we did, we'd expect some optical modes), and let the nearest neighbors of a lattice site be given by the set of vectors $\{\bv c\}$.

%%%%%%%%%
\end{document} 
